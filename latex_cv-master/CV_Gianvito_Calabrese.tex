\documentclass{muratcan_cv}
\usepackage{enumitem}

% !TeX program = lualatex

\setname{Gianvito}{Calabrese}
\setaddress{Avezzano/Italy}
\setmobile{(+39)3349499532}
\setmail{gcalabrese296@gmail.com}
\setposition{Work Student} %ignored for now
\setlinkedinaccount{https://www.linkedin.com/in/gianvito-calabrese/} %you can play with color of the template (red is also nice..)
\setgithubaccount{https://github.com/GianvitoCalabrese} %you can play with color of the template (red is also nice..)
\setthemecolor{red} %you can play with color of the template (red is also nice..)

\begin{document}
%Set variables
%You can add sections, texts, explanations just by copying the style below. Replace the dummy texts "\lipsum[1][x-x]\par" with actual texts.
%Create header
\headerview
\vspace{1ex}
%Sections
%
% Summary
\addblocktext{Summary}{%
I was born the 3rd of September 1992 in Bari where I spent my first 26 years. Afterwards I have settled in various places in Italy, everytime moved by the aim of better growth opportunities. I am a dedicated, creative and proactive physicist with three years job experience in the semiconductor related applications. I have designed novel devices and carried out experiments as academic technologist. I possess excellent analytical and communications skills and this allowed me to successfully move from university to industry in the role of yield enhancement engineer. I have spent my life always looking for opportunities to utilize my technical skills in a challenging working environment and become a valuable asset to the organization I work for.
{}}
%
%Education
\section{Education} 
    \datedexperience{University of Bari}{2011-2014} 
    \explanation{B.S in Physics} \\ [0.05cm]
     \explanationdetail{\coloredbullet\ % 
     Analog electronics, NIM modules, Theoretical Physics, Advanced
     Math.
     }
    \datedexperience{University of Bari}{2014-2017} 
    \explanation{M.S in Applied and Condensed Matter Physics} \\ [0.05cm]
    \explanationdetail{\coloredbullet\ %
      Material properties, Optoelectronic, Optical instruments, Semiconductors, Lasers, Digital electronics.
     }

% Experience
\section{Experience}
    %
    \datedexperience{IT Consultant}{Jen 2018 - Oct 2018    (Milan/Italy) } 
    \explanation{Exprivia S.p.a.} \\ [0.05cm]
    %
    \datedexperience{PhD student in Physics Engineering}{Nov 2018 - Dec 2019    (Como/Italy)} 
    \explanation{L-NESS Politecnico di Milano} 
    \begin{itemize}[noitemsep,topsep=0pt]
     \item \small Design and Fabrication of Graphene FET via eBeam Lithography. 
     \item Device Electrical Measurements and Characterization.
    \end{itemize} 
    %
    \datedexperience{Ext Manufacturing Yield Engineer}{Jen 2020 - on going    (Avezzano/Italy)} 
    \explanation{ON Semiconductor}
	 \begin{itemize}[noitemsep,topsep=0pt]
	\item \small Statistical analysis of wafer level probe data for yield enhancement purpose.
	\item Coordination and cooperation of the manufacturing process improvement activities at supplier level.
	\item Support on the production material management operations. 
	\item Basics of Product Engineering. 
	\end{itemize} 
    
\vspace{3ex}
% Skills
\section{Skills}
    %
    \newcommand{\skillone}{\createskill{Software}{\textbf{\emph{Experienced:}} \ \  Spotfire Exensio  \cpshalf \ \ JMP \ \ \textbf{\emph{Familiar:}} \ \  Orcad}}
    %
    \newcommand{\skilltwo}{\createskill{Programming Language}{\textbf{\emph{Familiar:}} \ \  Python \cpshalf \ \ SQL \ \cpshalf \ \ bash  \ \ \textbf{\emph{Occasional:}} \ \ C/C++ \cpshalf \ \ Mathematica}}
    %
    \newcommand{\skillthree}{\createskill{Operating System}{ Windows  \cpshalf \ \ GNU/Linux}}
    %
    \newcommand{\skillfour}{\createskill{Languages}{\textbf{\emph{Native:}} \ \  Italian \ \ \textbf{\emph{Fluent:}} \ \ English (*) }}
    %
    \newcommand{\skillfive}{\createskill{Driving License}{B}}
    \createskills{\skillone, \skilltwo, \skillthree, \skillfour, \skillfive}
%
% Experience
\section{Extra}
    \newcommand{\extraone}{%
   (*) Graded as B2 - based on TOEIC, Listening and Reading Test (27/09/2018)
    }
    %
    \newcommand{\extratwo}{%
    \lipsum[1][9-10]\par %replace this part with actual text
    }
    %
    \newcommand{\extrathree}{%
    \lipsum[1][11-12]%replace this part with actual text
    }
    %
    \newcommand{\listofextras}{\extraone}
    %
    \createbullets{\listofextras}
%
%Footnote
\createfootnote
\end{document}
